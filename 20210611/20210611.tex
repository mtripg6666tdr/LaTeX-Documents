\documentclass{jsarticle}
\usepackage{changepage,amsmath,amssymb,ascmac}
\usepackage{fancybox}
\usepackage[dvipdfmx]{graphicx}
\usepackage{indentfirst}
\newenvironment{indentblock}{\begin{adjustwidth}{\parindent}{}\hspace{-\parindent}}{\end{adjustwidth}}

\begin{document}
  \begin{screen}
    \ovalbox{問題} \\
    正の整数$a$, $b$, $c$, $d$について$a^3=b^2$, $c^3=d^2$, $c-a=9$ がなりたつとき、整数$a$,$b$,$c$,$d$の値を求めよ。
  \end{screen}
  \\
  \dotfill

  \ovalbox{解説} \\
  $a^3=b^2$より、$b=a^\frac{3}{2}=a\sqrt{a}$であるから、$b$が整数の時、$a$は平方数で表せる。\\
  \begin{itembox}[l]{なぜ$a$が平方数となるのか}
    まず$b$は問題文で定義されている通り整数で、$a$も同様に整数であるから、$\sqrt{a}$も整数となる。
    平方根が整数になるとき、$a$は平方数である。
    \[b=\sqrt{a} \iff b^2=a\]
  \end{itembox}
  $c$についても同様に平方数であるから、
  \[a=x^2, c=y^2\]
  と表せる。
  $c-a=9$より、$y^2-x^2=9$となる。
  また$c>a$であるから、$y>x$\\
  \[\therefore (y+x)(y-x)=9\] \\
  $x$、$y$は整数であるから、$y+x$、$y-x$も整数である。\\
  かつ、$y+x$、$y-x$はともに正
  ($\because y+x$は整数同士の足し算,$y-x$は$y>x$でかつ整数同士の引き算)\\
  $\therefore y+x>y-x$より、$ y+x$、$y-x$の考えられる組み合わせは、\\
  \[(y+x, y-x)=(9,1)\]
  したがって、\\
  \[x=5, y=4\]
  よって、\\
  \[a=25, c=16\]
  $a^3=b^2$, $c^3=d^2$により、\\
  \[b=125, d=64\]
  \\
  \ovalbox{解答}\\
  $a=25$\\
  $b=125$\\
  $c=16$\\
  $d=64$\\
\end{document}