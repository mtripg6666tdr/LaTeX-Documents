\documentclass{jsarticle}
\usepackage{amsmath,amssymb}
\usepackage{fancybox}
\begin{document}

99. 整式$x^3+ax^2+b=0$を$(x+2)^2$で割ると$4x+5$余るという。このとき、定数$a$,$b$の値を定めよ。\\
\\
$x^3+ax^2+b$を$(x+2)^2$で割った時の商を$x+c$とする。($\because x^3$の係数も$x^2$の係数も1)\\
このとき、次のように表せる。\\
$x^3+ax^2+b=(x+2)^2(x+c)+(4x+5)$\\
これを整理して、\\
$x^3+ax^2+b=x^3+(c+4)x^2+(4c+8)x+(4c+5)$\\
これを$x$に関する恒等式として係数を比較して、\\
$a=c+4$\\
$4c+8=0$\\
$b=4c+5$\\
したがって、\\
$\underline{ \ovalbox{解答} \ a=2,b=-3 \,(c=-2)}$

\end{document}
